% refer to https://en.wikibooks.org/wiki/LaTeX/Glossary for acronyms and glossary entries

\newacronym{AR}{AR}{augmented reality}
\newacronym{VR}{VR}{virtual reality}
\newacronym{HMD}{HMD}{head-mounted display}
\newacronym{HUD}{HUD}{head-up display}
\newacronym{FOV}{FOV}{field of view}
\newacronym{IMU}{IMU}{inertial measurement unit}
\newacronym{eVTOL}{eVTOL}{electric vertical take-off and landing aircraft}
\newacronym{WLT}{WLT}{World Locking Tools library}
\newacronym{CAD}{CAD}{computer aided design}
\newacronym{ICP}{ICP}{iterative closest point}
\newacronym[description={\glslink{enu}{East North Up}}]{ENU}{ENU}{East North Up}
\newacronym{ECEF}{ECEF}{Earth-centered, Earth-fixed}
\newacronym{CRS}{CRS}{coordinate reference system}
\newacronym{PEP}{PEP}{projection eye point}
\newacronym{GIS}{GIS}{geographic information system}
\newacronym[plural=GF-ELMes, longplural=geo-fixed external line meshes]{gfelm}{GF-ELM}{geo-fixed external line mesh}
\newacronym{TCAS}{TCAS}{traffic collision avoidance system}
\newacronym{API}{API}{application programming interface}
\newacronym{UDP}{UDP}{user datagram protocol}

\newglossaryentry{holoassist}
{
    name = {HoloAssist},
    description = {
        A Hololens application developed in Unity that allows to easily create augmented reality experiences to be shown in a fixed-platform flight simulator. It exposes a high-level API that can be used to draw \gls{geofixedaug} and \gls{planefixedaug}
    }
}

\newglossaryentry{holoassistapp}
{
    name = {HoloAssist App},
    description = {
        A script, written in an arbitrary programming language, that uses the \gls{holoassist} API to create some augmented reality experience
    }
}

\newglossaryentry{geofixedaug}
{
  name = {geo-fixed augmentation},
  description = {
      One of the two types of augmentations displayed by \gls{holoassist}. They consist in an arbitrary 3D mesh positioned at some geographical point. \gls{holoassist} does all the computation required to shown them in the correct geographical position when looking from the flight simulator cockpit's window
  },
}

\newglossaryentry{planefixedaug}
{
    name = {plane-fixed augmentation},
    description = {
        One of the two types of augmentations displayed by \gls{holoassist}. They consist in an arbitrary 3D mesh
        positioned at some fixed point inside the flight simulator cockpit
    }
}

\newglossaryentry{remoteunityeditor}
{
    name = {remote Unity editor},
    description = {
        A developer tool that allows to move and rotate a virtual 3D object in a Unity application running on the Hololens without having to recompile/redeploy the project. It also allows to retrieve the current pose of a \texttt{GameObject}
    }
}

\newglossaryentry{epsg4979}
{
    name = {EPSG:4979},
    description = {
        A commonly used \acrlong{CRS} based on latitude, longitude and altitude. See \autoref{section:geodesyfundamentals}
    }
}

\newglossaryentry{epsg4978}
{
    name = {EPSG:4978},
    description = {
        A commonly used \acrlong{ECEF} \acrlong{CRS}. See \autoref{section:geodesyfundamentals}
    }
}

\newglossaryentry{enu}
{
    name = {\glslink{ENU}{East North Up coordinate reference system}},
    description = {
        A commonly used topocentric \acrlong{CRS}. See \autoref{section:geodesyfundamentals}
    }
}

\newglossaryentry{planeENU}
{
    name = {airplane \gls{ENU} \gls{CRS}},
    description = {
        An \acrlong{ENU} \acrlong{CRS} centered at the current airplane's geographical position (as computed by the flight simulator)
    }
}