\chapter{\abstractname}

Flying an aircraft is a complex task, especially in abnormal situations. Although the cockpit environment has already been optimized to be as clear and effective as possible, there is the hypothesis that comparatively recent technologies like \gls{AR} could further improve the situation. Testing this hypothesis requires being able to quickly develop \gls{AR} experiences that can be used on innovative hardware like the Microsoft Hololens in the context of a flight simulator and then be evaluated by professional pilots.

Achieving this result requires a flexible and easy to use platform that simplifies the creation of these experiences: such platform is the main contribution of this thesis. Additionally, the developed framework has been used to create some \gls{AR} aids for landing at the Innsbruck airport, which have then been evaluated by real pilots on a fixed-platform aircraft simulator.

\makeatletter
\ifthenelse{\pdf@strcmp{\languagename}{english}=0}
{\renewcommand{\abstractname}{Kurzfassung}}
{\renewcommand{\abstractname}{Abstract}}
\makeatother